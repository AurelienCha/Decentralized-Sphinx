\usepackage{relsize} % for \mathlarger

% Colors tuto: https://www.overleaf.com/learn/latex/Using_colors_in_LaTeX
% \defincolor: {define new color}
%   - rgb: {r, g, b}                            [0, 1]
%   - RGB: {r, g, b}                            [0, 255]
%   - hsb: {hue, saturation, brightness}        [0, 1]
%   - HSB: {hue, saturation, brightness}        [360°, 100%, 100%]
%   - cmy: {cyan, magenta, yellow}              [0, 1]
%   - cmyk: {cyan, magenta, yellow, blacK}      [0, 1]
%   - gray: {gray}                              [0, 1] (low is dark)
%   - Gray: {Gray}                              [0, 15] (low is dark)
%   - HTML: {RRGGBB}                            6 hexa-digits
%   - wave: {wavelength}                        [363, 814] (nm)
% \colorlet: (i.e. mixing colors)
%   - \colorlet{mixedColor}{blue!10!red}        % 10% blue mixed with 90% red

\definecolor{unHighlight}{gray}{0.66}

\newcommand{\bigO}[1]{$\mathcal{O}(#1)$}

\newcommand{\mathtext}[1]{\text{#1}}
\newcommand{\indexsize}{\tiny}
\renewcommand{\index}[1]{\mathtext{\indexsize {#1}}}
\newcommand{\inmath}[1]{\relax\ifmmode #1 \else $#1$\fi}

\newcommand{\G}[1]{\inmath{G_\index{#1}}}
\newcommand{\N}[1]{\inmath{N_\index{#1}}}
\newcommand{\sk}[1]{\inmath{\text{sk}_\index{#1}}}
\newcommand{\PK}[1]{\inmath{\text{PK}_\index{#1}}}
\newcommand{\s}[1]{\inmath{s_\index{#1}}}
\renewcommand{\S}[1]{\inmath{S_\index{#1}}}

\newcommand{\sG}[2] {\relax
    \ifmmode \s{#1}\G{#2} 
    \else $\s{#1}\textcolor{unHighlight}{\G{#2}}$ 
\fi}
\newcommand{\B}[2]{\relax
    \ifmmode \mathlarger{\beta_\index{#1#2}} 
    \else $\mathlarger{\beta_\index{#1\textcolor{unHighlight}{#2}}}$ 
\fi}
\newcommand{\BB}[2]{\relax
    \ifmmode \mathlarger{\beta^\prime_\index{#1#2}} 
    \else $\mathlarger{\beta^\prime_\index{#1\textcolor{unHighlight}{#2}}}$ 
\fi}
\newcommand{\A}[1]{\inmath{\mathlarger{\mathlarger{\alpha_\index{#1}}}}}
\newcommand{\C}[1]{\inmath{\mathlarger{\mathlarger{\gamma_\index{#1}}}}}
\newcommand{\D}{\inmath{\mathlarger{\mathlarger{\mathrm{\Delta}}}}}



\newlength{\y}
\newcommand{\width}{10mm}
\newcommand{\height}{5.4mm}

\newcommand{\vgap}{
    \addtolength{\y}{-\height}
    \addtolength{\y}{-1mm}
}
\newcommand{\vGap}{
    \addtolength{\y}{-\height}
    \addtolength{\y}{-3mm}
}
%\newcommand{\hmac}{Eq. \ref{eq:integrity}}
\newcommand{\hmac}{\footnotesize $\sum$}

% Define TikZ styles
\usepackage{tikz}
\usetikzlibrary{shapes.geometric, positioning, calc, patterns, patterns.meta, backgrounds, fit, decorations.pathreplacing}
\tikzset{    
    % shape
    block/.style={draw, ultra thin, minimum width = #1*\width, minimum height = \height, inner sep = 0pt, anchor = west},
    inblock/.style={block = #1, densely dotted}, % dash pattern=on 1pt off 1pt, 
    zero_pad/.style={block = #1, fill = unHighlight, fill opacity=0.3},
    HMAC/.style={draw, ellipse, minimum width = \width, minimum height = \height, inner sep = 0pt, anchor = center},
    eq/.style={draw, circle, inner sep=1pt, fill opacity=0.2, fill=unHighlight, text opacity=1, text=black, font=\normalsize}, %font=\bfseries\large
    arrow/.style={->, shorten <= 1pt, shorten >= 2pt},
    line/.style={shorten <= 1pt, shorten >= 1pt},
    dotline/.style={dotted},
    % display
}

% Put tikz node in caption
\newcommand\eq[1][black]{ % Do not know why we need [black] to works and \eq instead of {\eq}
    \hspace{-2mm}
    \tikz[baseline=(char.base)]{
        \node [eq] (char) {#1};

    }
    \hspace{-2mm}
}