\section{Introduction}

Mix network, or mixnet, is an overlay network of servers (called mix nodes) that prevent an adversary from correlating senders with receivers~\cite{chaum-mix,cypherpunk-remailer,piotrowska2017loopix,nym-network-whitepaper,danezis2003mixminion, van2015vuvuzela,mixmaster-spec,chaum2016cmix}. They achieve this by adding delays to messages inside the different mixnodes in order to mitigate timing analysis attacks. Additionally, unlike Tor~\cite{onion-routing96} where all packets sent by a user follow the same path for the entire session (called circuit-based), in mixnets routing is typically packet-based (meaning that each packet follows a different path). These two techniques ensure that mixnets are resilient against a strong adversary who observes the entire input and outputs of the network typically called a Global Passive Adversary (GPA).
During the last few decades several systems have been proposed in the literature and few of them have been implemented. Open research problems such as parameterization, authentication and fake dummies have halted the usage on a large scale. Recently, Nym Technologies a mixnet-based company is being commercialized and proposing a mixnet network for services to be integrated with their network for a fee. Their network is based on Loopix~\cite{piotrowska2017loopix}. Users of these services are then allowed to use the Nym mixnet by using Nym credentials (based on the Coconut credential~\cite{coconut}) that they can construct after getting a certificate of paying for a specific service to use the Nym network. However, nothing prevents users from cheating who might exploit a valid Nym credential for another service they did not pay for. This is a particular difficult problem in anonymous communication networks where the mixnodes do not know the traffic type or the final service a user is communicating with by doing layered encryption where the final IP address is only know by the last node in the path using a packer format such as Sphinx~\cite{sphinx}. 

In this paper we present a scheme that creates the Sphinx header in a
decentralized way based on trusted third parties while ensuring that these
parties learn nothing about the destination or the path. Our work makes the
following contributions:
%
\begin{itemize}
%
  \item{\ldots}
%
  \item{\ldots}
%
  \item{We assess the impact on availability and economic viability
adversaries can have in Sphinx and our solution and conclude that\ldots}
%
\end{itemize}
%
\todo{Something on prototype and availability of artifacts!}
\todo{For CBT: highlight relevance of the contribution to Lightning and Nym
in intro and contributions.}

We highlight related work and motivation in Section~\ref{sec:related}, then we specify and justify our system model in Section~\ref{sec:sys_model}, where we also describe the considered threat model. We then present our scheme that decentralize the creation of the Sphinx headers in Section~\ref{sec:scheme} and the evaluation of our proposed solution in Section~\ref{sec:eval}. Finally we conclude and discuss future work in Section~\ref{sec:conclusion}
