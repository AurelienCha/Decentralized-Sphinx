\section{Motivation and Related Work}
\label{sec:related}

Since Chaum’s seminal work on untraceable email in 1981~\cite{chaum-mix}, there has been a great amount of research related to mixnets' design~\cite{piotrowska2017loopix, van2015vuvuzela, kwon2020xrd, lazar2018karaoke, cottrell1995mixmaster, alexopoulos2017MCMIX, chaum2016cmix, chaum-mix, danezis2003mixminion}. However most systems that have been deployed and used come with their own system on top of the network meaning that only one type of traffic type is allowed. As beautifully stated by Dingledine et al. in~\cite{dingledine2006anonymity}, "Anonymity Loves Company" meaning that the more messages there are in the network the more privacy the network provide. This is also shared by Ben Guirat et al. in~\cite{benguirat2023blending} where the authors show that blending different traffic types on top of a mixnet provide better anonymity, meaning let's imagine an instant messaging system where users do not tolerate latency of more than few seconds and an email app where users tolerate latency of up to 1 minute. The authors show that blending these two types of traffic do actually increase the privacy for both traffic. This only applicable in networks such as Tor or the Nym network that they offer the network for different applications to be integrated on top of the network rather than dictating which application to use the mixnet.
However, certain open research problems remain open. For example how can we ensure that certain traffic are not allowed (for whatever reason and we will specify the exact reason for our work) without compromising ?
Tor solves the problem with having exit policy that simply drop traffic at the last node. However Tor routing is circuit-based meaning that all traffic from a specefic user session follow one path and hence packets are encrypted using symetric cryptographhy which is cheap, so even if traffic is dropped that is not a problem.

However unlike Tor, mixnets can be expensive due to the routing type which
is packet based, meaning that every packet takes a different path and
hence layered encrypted using public key cryptography. This means that if
packets are dropped at the last nodes when not allowed, this waste a huge
amount of cryptographic power. Additionally unlike Tor where relay are
volunteer based, nodes in mixnets such as Nym are economically incentivized
based on the bandwidth they routed so if packets are dropped at the end
this is not great.
%
\todo{Iness: add a simple graph here of a mixnet}
