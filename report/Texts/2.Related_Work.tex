\section{Motivation and Related Work}
\label{sec:related}

Since Chaum’s seminal work on untraceable email in 1981~\cite{chaum-mix}, there has been a great amount of research related to mixnets' design~\cite{piotrowska2017loopix, van2015vuvuzela, kwon2020xrd, lazar2018karaoke, cottrell1995mixmaster, alexopoulos2017MCMIX, chaum2016cmix, chaum-mix, danezis2003mixminion}. However most systems that have been deployed and used come with their own system on top of the network meaning that only one type of traffic type is allowed. As beautifully stated by Dingledine et al. in~\cite{dingledine2006anonymity}, "Anonymity Loves Company" meaning that the more messages there are in the network the more privacy the network provide. This is also shared by Ben Guirat et al. in~\cite{benguirat2023blending} where the authors show that blending different traffic types on top of a mixnet provide better anonymity, meaning let's imagine an instant messaging system where users do not tolerate latency of more than few seconds and an email app where users tolerate latency of up to 1 minute. The authors show that blending these two types of traffic do actually increase the privacy for both traffic. This only applicable in networks such as Tor or the Nym network that they offer the network for different applications to be integrated on top of the network rather than dictating which application to use the mixnet.
However, certain open research problems remain open. For example how can we ensure that certain traffic are not allowed (for whatever reason and we will specify the exact reason for our work) without compromising ?
Tor solves the problem with having exit policy that simply drop traffic at the last node. However Tor routing is circuit-based meaning that all traffic from a specific user session follow one path and hence packets are encrypted using symmetric cryptography which is cheap, so even if traffic is dropped that is not a problem.

However, and beyond Tor, mixnets can impose substantial computational
overheads due to packet based routing where every packet takes a different
path and hence layered public-key cryptography is used. This implies that
that, if packets are dropped at the last node of the mix, which is
traditionally the only place where service contracts can be validated,
computational resources required by the mix to propagate a packet to its
final node are wasted. Additionally, and unlike Tor where relaying traffic
happens on a volunteering base, nodes in mixnets such as Nym are
economically incentivized through payment per relayed bandwidth. Thus,
dropping packets at an the exit node impacts the economic viability of the
mixnet service.

\begin{questions}
%
        \item \ldots
%
\end{questions}
%
\todo{Some concrete research questions that we can derive here and answer?}




